\documentclass{article}

% Language setting
% Replace `english' with e.g. `spanish' to change the document language
\usepackage[english]{babel}

% Set page size and margins
% Replace `letterpaper' with `a4paper' for UK/EU standard size
\usepackage[letterpaper,top=2cm,bottom=2cm,left=3cm,right=3cm,marginparwidth=1.75cm]{geometry}

% Useful packages
\usepackage{amsmath}
\usepackage{graphicx}
\usepackage[colorlinks=true, allcolors=blue]{hyperref}

\title{Haocheng and Data science}
\author{Haocheng Yang}

\begin{document}
\maketitle

\begin{abstract}
This paper summarizes Haocheng's interests in the course that is entitled data science for economists. In addition, Haocheng shares some of his hobbies and future career goals in this paper.
\end{abstract}

\section{Introduction}

Nowadays, data science has become an indispensable part of industry and academia. As a fresh-new accounting Ph.D. student, I need to acquire necessary knowledge to use advanced and efficient analytical tools to conduct my own empirical research. Data science for economists is a course as fit for my expectation as a pudding for a friar's mouth. According to the syllable of this course, we will be involved in using different coding languages and doing some machine learning projects. I prefer to use python as my daily data analysis tool, but I can acquire fundamental ideas using other languages, like R, in this class. As for course content, I am especially interested in fundamental machine learning, how to train the data analysis model and how to interchangeably use different coding languages in an efficient way. So far, I don't have a specific idea about the project for the class, but I have a conceptual framework in my mind which is whether the more similar disclosure content among different firms signal the lower effort firms made or lower disclosure quality to convey information?. My goal is to be a professor of accounting after graduation.

\section{Equation}
\(x^2 + y^2 = z^2\)

\end{document}